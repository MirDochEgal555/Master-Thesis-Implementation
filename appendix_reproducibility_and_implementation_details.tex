% Suggested inclusion in main thesis file:
% \appendix
% % Suggested inclusion in main thesis file:
% \appendix
% % Suggested inclusion in main thesis file:
% \appendix
% % Suggested inclusion in main thesis file:
% \appendix
% \input{appendix_reproducibility_and_implementation_details}

\chapter{Reproducibility and Implementation Details}
\label{app:reproducibility}

This appendix documents the implementation and experiment protocol used in the
repository \texttt{Master-Thesis-Implementation}. The goal is to make each
reported result reproducible from source code, configuration defaults, and
generated artifacts.

\section{Codebase and Entry Points}

Experiments are organized around the following entry scripts:
\begin{itemize}
    \item \texttt{main.py}: single-run training and backtest.
    \item \texttt{run\_grid\_search.py}: parallel grid search over seeds and hyperparameters.
    \item \texttt{run\_backtest\_grid.py}: parallel backtest grid over seeds.
    \item \texttt{sweep\_grid.py}: extended sweep utility with incremental CSV logging.
    \item \texttt{baselines.py}: baseline portfolio strategies.
    \item \texttt{grid\_search\_report.py}: parsing and aggregation of line logs.
    \item \texttt{plot\_learning\_stats.py}, \texttt{learning\_comparison.py}: plotting utilities.
\end{itemize}

Core logic is implemented in the \texttt{portfolio\_rl/} package:
data loading, model definitions, Kalman filtering, dynamics model, rollouts,
training, and evaluation.

\section{Software Environment}

\subsection{Dependencies}

All Python dependencies are pinned in \texttt{requirements.txt}. Key packages
used in experiments are:
\texttt{torch}, \texttt{numpy}, \texttt{pandas}, \texttt{scipy},
\texttt{matplotlib}, \texttt{yfinance}, and \texttt{pyarrow}.

\subsection{Environment Setup}

\begin{verbatim}
python -m venv .venv
# Windows PowerShell
.venv\Scripts\Activate.ps1
# macOS/Linux
source .venv/bin/activate
pip install -r requirements.txt
\end{verbatim}

\section{Data, Universe, and Splits}

Data is downloaded with \texttt{yfinance} through
\texttt{portfolio\_rl.data.YahooReturnsDataset}. Price data is converted to
daily simple returns via percentage change and cached to
\texttt{returns.parquet}.

\begin{itemize}
    \item Asset universe (all default run scripts): \{MSFT, JPM, JNJ, XOM, PG\}.
    \item Date range: 2022-01-01 to 2024-12-31.
    \item Price field: \texttt{Close}.
    \item Train/validation/test split:
    \begin{itemize}
        \item train: dates $\le$ 2023-03-24
        \item validation: 2023-03-25 to 2023-09-30
        \item test: dates $\ge$ 2023-10-01
    \end{itemize}
\end{itemize}

Expanding covariance matrices are computed per split with optional diagonal
projection for training.

\section{Model and Training Implementation}

\subsection{Policy, Value, and Dynamics Networks}

\begin{itemize}
    \item Policy network (\texttt{PolicyNet}): MLP with two hidden layers and
    ReLU activations, outputting logits for a softmax portfolio.
    \item Value network (\texttt{ValueNet}): MLP with two hidden layers and a
    scalar output.
    \item Dynamics model (\texttt{DynamicsModel}): MLP on concatenated
    state-action pairs, predicting Gaussian mean and diagonal variance of the
    next state.
\end{itemize}

The hidden width is controlled by script-level settings (e.g.,
\texttt{networksize} in \texttt{main.py} and \texttt{run\_grid\_search.py}).

\subsection{Kalman Filter}

The filter implementation (\texttt{LearnableKalman}) uses:
\begin{itemize}
    \item fixed transition matrix $A = I$;
    \item diagonal process and measurement noise ($Q$, $R$);
    \item either learned log-diagonal parameters or fixed values provided in
    ablations.
\end{itemize}

\subsection{Reward and Objective}

At each step, the Markowitz-style component is:
\[
r^{\text{mark}}_t
= w_{t-1}^\top z_t
- \lambda \, w_{t-1}^\top \Sigma_t w_{t-1}.
\]

Optionally, uncertainty, turnover cost, and turnover-shaping terms are added:
\[
r_t
= r^{\text{mark}}_t
- \kappa_{\text{unc}} \cdot \tfrac{1}{2}\log\det(\Sigma^{\text{unc}}_t)
- c \lVert w_t - w_{t-1} \rVert_1
+ \beta \exp\!\left(-\tfrac{3}{2}
\left(\frac{\lVert w_t - w_{t-1} \rVert_1}{\tau} - 1\right)^2\right).
\]

Training combines actor, critic, Kalman, dynamics, and simulated policy losses
with scalar weights from \texttt{TrainConfig}.

\subsection{Training Phases}

The main training loop in \texttt{portfolio\_rl/backtest\_seed.py} uses:
\begin{itemize}
    \item 100 epochs KF warmup (\texttt{warmup\_kf}),
    \item 100 epochs dynamics warmup (\texttt{warmup\_dyn}),
    \item policy training for \texttt{cfg.updates} epochs (\texttt{train}).
\end{itemize}

Windows are iterated with
\texttt{T = window\_size + 5} and \texttt{stride = T}. Validation Sharpe is
evaluated periodically using \texttt{cfg.print\_every}.

\section{Evaluation Protocol}

The full-run evaluator computes portfolio returns from lagged weights and
reports:
\begin{itemize}
    \item Sharpe ratio:
    $\text{Sharpe} = \frac{\bar{r}}{s_r + 10^{-8}}\sqrt{252}$,
    \item total return proxy: $1 + \sum_t r_t$,
    \item mean and standard deviation of per-step returns,
    \item maximum drawdown from cumulative equity.
\end{itemize}

During evaluation, policy output is converted to long-only weights via softmax,
with a small $\epsilon$-smoothing and renormalization step.

\section{Determinism and Parallel Execution}

For each run, seeds are set for Python, NumPy, and PyTorch using
\texttt{set\_seed(seed)}. Parallel workers enforce one-thread execution via:
\texttt{torch.set\_num\_threads(1)},
\texttt{OMP\_NUM\_THREADS=1}, and
\texttt{MKL\_NUM\_THREADS=1}.

In spite of seed control, exact bitwise reproducibility is not guaranteed
across machines or library versions because of floating-point and backend
differences.

\section{Default Script Configurations}

The defaults currently in the repository are:
\begin{itemize}
    \item \texttt{main.py}: single run (seed 0), \texttt{window\_size=1},
    \texttt{lam=10}, \texttt{networksize=128}, \texttt{learnrate=1e-5},
    \texttt{updates=1500}, with optional best-checkpoint saving.
    \item \texttt{run\_grid\_search.py}: seeds 0--9, currently configured with
    one-point grids for window size, lambda, network size, and learning rate.
    \item \texttt{run\_backtest\_grid.py}: seeds 0--20, currently configured
    with one-point grids for window size and lambda.
    \item \texttt{sweep\_grid.py}: configurable wider search with append-safe
    CSV logging in \texttt{sweep\_results.csv}.
\end{itemize}

These scripts are intentionally parameterized directly in code. Reproducing
specific thesis figures should therefore use the exact committed values from
the corresponding experiment revision.

\section{Artifacts and Result Files}

Typical generated files are:
\begin{itemize}
    \item \texttt{returns.parquet}: cached returns data.
    \item \texttt{learning\_stats\_kf.csv}: per-epoch training statistics.
    \item \texttt{grid\_search\_lines.txt}: per-job grid log.
    \item \texttt{grid\_search\_summary.txt}: aggregated grid summary.
    \item \texttt{backtest\_grid.txt}: aggregated backtest summary.
    \item \texttt{weights.txt}: latest saved test weights.
    \item \texttt{grid\_search\_raw.csv}, \texttt{grid\_search\_seed\_avg.csv},
    \texttt{grid\_search\_lambda\_avg.csv},
    \texttt{grid\_search\_window\_avg.csv},
    \texttt{grid\_search\_networksize\_avg.csv},
    \texttt{grid\_search\_learnrate\_avg.csv}.
\end{itemize}

The report CSVs are produced by:
\begin{verbatim}
python grid_search_report.py
\end{verbatim}

\section{Minimal Reproduction Procedure}

\begin{enumerate}
    \item Create and activate a clean Python environment.
    \item Install dependencies from \texttt{requirements.txt}.
    \item Run one of:
    \begin{verbatim}
python main.py
python run_grid_search.py
python run_backtest_grid.py
\end{verbatim}
    \item For grid-search post-processing:
    \begin{verbatim}
python grid_search_report.py
\end{verbatim}
    \item For plots:
    \begin{verbatim}
python plot_learning_stats.py learning_stats_kf.csv --out-dir .
\end{verbatim}
\end{enumerate}

This workflow reproduces the full training, evaluation, and artifact-generation
pipeline implemented in the repository.


\chapter{Reproducibility and Implementation Details}
\label{app:reproducibility}

This appendix documents the implementation and experiment protocol used in the
repository \texttt{Master-Thesis-Implementation}. The goal is to make each
reported result reproducible from source code, configuration defaults, and
generated artifacts.

\section{Codebase and Entry Points}

Experiments are organized around the following entry scripts:
\begin{itemize}
    \item \texttt{main.py}: single-run training and backtest.
    \item \texttt{run\_grid\_search.py}: parallel grid search over seeds and hyperparameters.
    \item \texttt{run\_backtest\_grid.py}: parallel backtest grid over seeds.
    \item \texttt{sweep\_grid.py}: extended sweep utility with incremental CSV logging.
    \item \texttt{baselines.py}: baseline portfolio strategies.
    \item \texttt{grid\_search\_report.py}: parsing and aggregation of line logs.
    \item \texttt{plot\_learning\_stats.py}, \texttt{learning\_comparison.py}: plotting utilities.
\end{itemize}

Core logic is implemented in the \texttt{portfolio\_rl/} package:
data loading, model definitions, Kalman filtering, dynamics model, rollouts,
training, and evaluation.

\section{Software Environment}

\subsection{Dependencies}

All Python dependencies are pinned in \texttt{requirements.txt}. Key packages
used in experiments are:
\texttt{torch}, \texttt{numpy}, \texttt{pandas}, \texttt{scipy},
\texttt{matplotlib}, \texttt{yfinance}, and \texttt{pyarrow}.

\subsection{Environment Setup}

\begin{verbatim}
python -m venv .venv
# Windows PowerShell
.venv\Scripts\Activate.ps1
# macOS/Linux
source .venv/bin/activate
pip install -r requirements.txt
\end{verbatim}

\section{Data, Universe, and Splits}

Data is downloaded with \texttt{yfinance} through
\texttt{portfolio\_rl.data.YahooReturnsDataset}. Price data is converted to
daily simple returns via percentage change and cached to
\texttt{returns.parquet}.

\begin{itemize}
    \item Asset universe (all default run scripts): \{MSFT, JPM, JNJ, XOM, PG\}.
    \item Date range: 2022-01-01 to 2024-12-31.
    \item Price field: \texttt{Close}.
    \item Train/validation/test split:
    \begin{itemize}
        \item train: dates $\le$ 2023-03-24
        \item validation: 2023-03-25 to 2023-09-30
        \item test: dates $\ge$ 2023-10-01
    \end{itemize}
\end{itemize}

Expanding covariance matrices are computed per split with optional diagonal
projection for training.

\section{Model and Training Implementation}

\subsection{Policy, Value, and Dynamics Networks}

\begin{itemize}
    \item Policy network (\texttt{PolicyNet}): MLP with two hidden layers and
    ReLU activations, outputting logits for a softmax portfolio.
    \item Value network (\texttt{ValueNet}): MLP with two hidden layers and a
    scalar output.
    \item Dynamics model (\texttt{DynamicsModel}): MLP on concatenated
    state-action pairs, predicting Gaussian mean and diagonal variance of the
    next state.
\end{itemize}

The hidden width is controlled by script-level settings (e.g.,
\texttt{networksize} in \texttt{main.py} and \texttt{run\_grid\_search.py}).

\subsection{Kalman Filter}

The filter implementation (\texttt{LearnableKalman}) uses:
\begin{itemize}
    \item fixed transition matrix $A = I$;
    \item diagonal process and measurement noise ($Q$, $R$);
    \item either learned log-diagonal parameters or fixed values provided in
    ablations.
\end{itemize}

\subsection{Reward and Objective}

At each step, the Markowitz-style component is:
\[
r^{\text{mark}}_t
= w_{t-1}^\top z_t
- \lambda \, w_{t-1}^\top \Sigma_t w_{t-1}.
\]

Optionally, uncertainty, turnover cost, and turnover-shaping terms are added:
\[
r_t
= r^{\text{mark}}_t
- \kappa_{\text{unc}} \cdot \tfrac{1}{2}\log\det(\Sigma^{\text{unc}}_t)
- c \lVert w_t - w_{t-1} \rVert_1
+ \beta \exp\!\left(-\tfrac{3}{2}
\left(\frac{\lVert w_t - w_{t-1} \rVert_1}{\tau} - 1\right)^2\right).
\]

Training combines actor, critic, Kalman, dynamics, and simulated policy losses
with scalar weights from \texttt{TrainConfig}.

\subsection{Training Phases}

The main training loop in \texttt{portfolio\_rl/backtest\_seed.py} uses:
\begin{itemize}
    \item 100 epochs KF warmup (\texttt{warmup\_kf}),
    \item 100 epochs dynamics warmup (\texttt{warmup\_dyn}),
    \item policy training for \texttt{cfg.updates} epochs (\texttt{train}).
\end{itemize}

Windows are iterated with
\texttt{T = window\_size + 5} and \texttt{stride = T}. Validation Sharpe is
evaluated periodically using \texttt{cfg.print\_every}.

\section{Evaluation Protocol}

The full-run evaluator computes portfolio returns from lagged weights and
reports:
\begin{itemize}
    \item Sharpe ratio:
    $\text{Sharpe} = \frac{\bar{r}}{s_r + 10^{-8}}\sqrt{252}$,
    \item total return proxy: $1 + \sum_t r_t$,
    \item mean and standard deviation of per-step returns,
    \item maximum drawdown from cumulative equity.
\end{itemize}

During evaluation, policy output is converted to long-only weights via softmax,
with a small $\epsilon$-smoothing and renormalization step.

\section{Determinism and Parallel Execution}

For each run, seeds are set for Python, NumPy, and PyTorch using
\texttt{set\_seed(seed)}. Parallel workers enforce one-thread execution via:
\texttt{torch.set\_num\_threads(1)},
\texttt{OMP\_NUM\_THREADS=1}, and
\texttt{MKL\_NUM\_THREADS=1}.

In spite of seed control, exact bitwise reproducibility is not guaranteed
across machines or library versions because of floating-point and backend
differences.

\section{Default Script Configurations}

The defaults currently in the repository are:
\begin{itemize}
    \item \texttt{main.py}: single run (seed 0), \texttt{window\_size=1},
    \texttt{lam=10}, \texttt{networksize=128}, \texttt{learnrate=1e-5},
    \texttt{updates=1500}, with optional best-checkpoint saving.
    \item \texttt{run\_grid\_search.py}: seeds 0--9, currently configured with
    one-point grids for window size, lambda, network size, and learning rate.
    \item \texttt{run\_backtest\_grid.py}: seeds 0--20, currently configured
    with one-point grids for window size and lambda.
    \item \texttt{sweep\_grid.py}: configurable wider search with append-safe
    CSV logging in \texttt{sweep\_results.csv}.
\end{itemize}

These scripts are intentionally parameterized directly in code. Reproducing
specific thesis figures should therefore use the exact committed values from
the corresponding experiment revision.

\section{Artifacts and Result Files}

Typical generated files are:
\begin{itemize}
    \item \texttt{returns.parquet}: cached returns data.
    \item \texttt{learning\_stats\_kf.csv}: per-epoch training statistics.
    \item \texttt{grid\_search\_lines.txt}: per-job grid log.
    \item \texttt{grid\_search\_summary.txt}: aggregated grid summary.
    \item \texttt{backtest\_grid.txt}: aggregated backtest summary.
    \item \texttt{weights.txt}: latest saved test weights.
    \item \texttt{grid\_search\_raw.csv}, \texttt{grid\_search\_seed\_avg.csv},
    \texttt{grid\_search\_lambda\_avg.csv},
    \texttt{grid\_search\_window\_avg.csv},
    \texttt{grid\_search\_networksize\_avg.csv},
    \texttt{grid\_search\_learnrate\_avg.csv}.
\end{itemize}

The report CSVs are produced by:
\begin{verbatim}
python grid_search_report.py
\end{verbatim}

\section{Minimal Reproduction Procedure}

\begin{enumerate}
    \item Create and activate a clean Python environment.
    \item Install dependencies from \texttt{requirements.txt}.
    \item Run one of:
    \begin{verbatim}
python main.py
python run_grid_search.py
python run_backtest_grid.py
\end{verbatim}
    \item For grid-search post-processing:
    \begin{verbatim}
python grid_search_report.py
\end{verbatim}
    \item For plots:
    \begin{verbatim}
python plot_learning_stats.py learning_stats_kf.csv --out-dir .
\end{verbatim}
\end{enumerate}

This workflow reproduces the full training, evaluation, and artifact-generation
pipeline implemented in the repository.


\chapter{Reproducibility and Implementation Details}
\label{app:reproducibility}

This appendix documents the implementation and experiment protocol used in the
repository \texttt{Master-Thesis-Implementation}. The goal is to make each
reported result reproducible from source code, configuration defaults, and
generated artifacts.

\section{Codebase and Entry Points}

Experiments are organized around the following entry scripts:
\begin{itemize}
    \item \texttt{main.py}: single-run training and backtest.
    \item \texttt{run\_grid\_search.py}: parallel grid search over seeds and hyperparameters.
    \item \texttt{run\_backtest\_grid.py}: parallel backtest grid over seeds.
    \item \texttt{sweep\_grid.py}: extended sweep utility with incremental CSV logging.
    \item \texttt{baselines.py}: baseline portfolio strategies.
    \item \texttt{grid\_search\_report.py}: parsing and aggregation of line logs.
    \item \texttt{plot\_learning\_stats.py}, \texttt{learning\_comparison.py}: plotting utilities.
\end{itemize}

Core logic is implemented in the \texttt{portfolio\_rl/} package:
data loading, model definitions, Kalman filtering, dynamics model, rollouts,
training, and evaluation.

\section{Software Environment}

\subsection{Dependencies}

All Python dependencies are pinned in \texttt{requirements.txt}. Key packages
used in experiments are:
\texttt{torch}, \texttt{numpy}, \texttt{pandas}, \texttt{scipy},
\texttt{matplotlib}, \texttt{yfinance}, and \texttt{pyarrow}.

\subsection{Environment Setup}

\begin{verbatim}
python -m venv .venv
# Windows PowerShell
.venv\Scripts\Activate.ps1
# macOS/Linux
source .venv/bin/activate
pip install -r requirements.txt
\end{verbatim}

\section{Data, Universe, and Splits}

Data is downloaded with \texttt{yfinance} through
\texttt{portfolio\_rl.data.YahooReturnsDataset}. Price data is converted to
daily simple returns via percentage change and cached to
\texttt{returns.parquet}.

\begin{itemize}
    \item Asset universe (all default run scripts): \{MSFT, JPM, JNJ, XOM, PG\}.
    \item Date range: 2022-01-01 to 2024-12-31.
    \item Price field: \texttt{Close}.
    \item Train/validation/test split:
    \begin{itemize}
        \item train: dates $\le$ 2023-03-24
        \item validation: 2023-03-25 to 2023-09-30
        \item test: dates $\ge$ 2023-10-01
    \end{itemize}
\end{itemize}

Expanding covariance matrices are computed per split with optional diagonal
projection for training.

\section{Model and Training Implementation}

\subsection{Policy, Value, and Dynamics Networks}

\begin{itemize}
    \item Policy network (\texttt{PolicyNet}): MLP with two hidden layers and
    ReLU activations, outputting logits for a softmax portfolio.
    \item Value network (\texttt{ValueNet}): MLP with two hidden layers and a
    scalar output.
    \item Dynamics model (\texttt{DynamicsModel}): MLP on concatenated
    state-action pairs, predicting Gaussian mean and diagonal variance of the
    next state.
\end{itemize}

The hidden width is controlled by script-level settings (e.g.,
\texttt{networksize} in \texttt{main.py} and \texttt{run\_grid\_search.py}).

\subsection{Kalman Filter}

The filter implementation (\texttt{LearnableKalman}) uses:
\begin{itemize}
    \item fixed transition matrix $A = I$;
    \item diagonal process and measurement noise ($Q$, $R$);
    \item either learned log-diagonal parameters or fixed values provided in
    ablations.
\end{itemize}

\subsection{Reward and Objective}

At each step, the Markowitz-style component is:
\[
r^{\text{mark}}_t
= w_{t-1}^\top z_t
- \lambda \, w_{t-1}^\top \Sigma_t w_{t-1}.
\]

Optionally, uncertainty, turnover cost, and turnover-shaping terms are added:
\[
r_t
= r^{\text{mark}}_t
- \kappa_{\text{unc}} \cdot \tfrac{1}{2}\log\det(\Sigma^{\text{unc}}_t)
- c \lVert w_t - w_{t-1} \rVert_1
+ \beta \exp\!\left(-\tfrac{3}{2}
\left(\frac{\lVert w_t - w_{t-1} \rVert_1}{\tau} - 1\right)^2\right).
\]

Training combines actor, critic, Kalman, dynamics, and simulated policy losses
with scalar weights from \texttt{TrainConfig}.

\subsection{Training Phases}

The main training loop in \texttt{portfolio\_rl/backtest\_seed.py} uses:
\begin{itemize}
    \item 100 epochs KF warmup (\texttt{warmup\_kf}),
    \item 100 epochs dynamics warmup (\texttt{warmup\_dyn}),
    \item policy training for \texttt{cfg.updates} epochs (\texttt{train}).
\end{itemize}

Windows are iterated with
\texttt{T = window\_size + 5} and \texttt{stride = T}. Validation Sharpe is
evaluated periodically using \texttt{cfg.print\_every}.

\section{Evaluation Protocol}

The full-run evaluator computes portfolio returns from lagged weights and
reports:
\begin{itemize}
    \item Sharpe ratio:
    $\text{Sharpe} = \frac{\bar{r}}{s_r + 10^{-8}}\sqrt{252}$,
    \item total return proxy: $1 + \sum_t r_t$,
    \item mean and standard deviation of per-step returns,
    \item maximum drawdown from cumulative equity.
\end{itemize}

During evaluation, policy output is converted to long-only weights via softmax,
with a small $\epsilon$-smoothing and renormalization step.

\section{Determinism and Parallel Execution}

For each run, seeds are set for Python, NumPy, and PyTorch using
\texttt{set\_seed(seed)}. Parallel workers enforce one-thread execution via:
\texttt{torch.set\_num\_threads(1)},
\texttt{OMP\_NUM\_THREADS=1}, and
\texttt{MKL\_NUM\_THREADS=1}.

In spite of seed control, exact bitwise reproducibility is not guaranteed
across machines or library versions because of floating-point and backend
differences.

\section{Default Script Configurations}

The defaults currently in the repository are:
\begin{itemize}
    \item \texttt{main.py}: single run (seed 0), \texttt{window\_size=1},
    \texttt{lam=10}, \texttt{networksize=128}, \texttt{learnrate=1e-5},
    \texttt{updates=1500}, with optional best-checkpoint saving.
    \item \texttt{run\_grid\_search.py}: seeds 0--9, currently configured with
    one-point grids for window size, lambda, network size, and learning rate.
    \item \texttt{run\_backtest\_grid.py}: seeds 0--20, currently configured
    with one-point grids for window size and lambda.
    \item \texttt{sweep\_grid.py}: configurable wider search with append-safe
    CSV logging in \texttt{sweep\_results.csv}.
\end{itemize}

These scripts are intentionally parameterized directly in code. Reproducing
specific thesis figures should therefore use the exact committed values from
the corresponding experiment revision.

\section{Artifacts and Result Files}

Typical generated files are:
\begin{itemize}
    \item \texttt{returns.parquet}: cached returns data.
    \item \texttt{learning\_stats\_kf.csv}: per-epoch training statistics.
    \item \texttt{grid\_search\_lines.txt}: per-job grid log.
    \item \texttt{grid\_search\_summary.txt}: aggregated grid summary.
    \item \texttt{backtest\_grid.txt}: aggregated backtest summary.
    \item \texttt{weights.txt}: latest saved test weights.
    \item \texttt{grid\_search\_raw.csv}, \texttt{grid\_search\_seed\_avg.csv},
    \texttt{grid\_search\_lambda\_avg.csv},
    \texttt{grid\_search\_window\_avg.csv},
    \texttt{grid\_search\_networksize\_avg.csv},
    \texttt{grid\_search\_learnrate\_avg.csv}.
\end{itemize}

The report CSVs are produced by:
\begin{verbatim}
python grid_search_report.py
\end{verbatim}

\section{Minimal Reproduction Procedure}

\begin{enumerate}
    \item Create and activate a clean Python environment.
    \item Install dependencies from \texttt{requirements.txt}.
    \item Run one of:
    \begin{verbatim}
python main.py
python run_grid_search.py
python run_backtest_grid.py
\end{verbatim}
    \item For grid-search post-processing:
    \begin{verbatim}
python grid_search_report.py
\end{verbatim}
    \item For plots:
    \begin{verbatim}
python plot_learning_stats.py learning_stats_kf.csv --out-dir .
\end{verbatim}
\end{enumerate}

This workflow reproduces the full training, evaluation, and artifact-generation
pipeline implemented in the repository.


\chapter{Reproducibility and Implementation Details}
\label{app:reproducibility}

This appendix documents the implementation and experiment protocol used in the
repository \texttt{Master-Thesis-Implementation}. The goal is to make each
reported result reproducible from source code, configuration defaults, and
generated artifacts.

\section{Codebase and Entry Points}

Experiments are organized around the following entry scripts:
\begin{itemize}
    \item \texttt{main.py}: single-run training and backtest.
    \item \texttt{run\_grid\_search.py}: parallel grid search over seeds and hyperparameters.
    \item \texttt{run\_backtest\_grid.py}: parallel backtest grid over seeds.
    \item \texttt{sweep\_grid.py}: extended sweep utility with incremental CSV logging.
    \item \texttt{baselines.py}: baseline portfolio strategies.
    \item \texttt{grid\_search\_report.py}: parsing and aggregation of line logs.
    \item \texttt{plot\_learning\_stats.py}, \texttt{learning\_comparison.py}: plotting utilities.
\end{itemize}

Core logic is implemented in the \texttt{portfolio\_rl/} package:
data loading, model definitions, Kalman filtering, dynamics model, rollouts,
training, and evaluation.

\section{Software Environment}

\subsection{Dependencies}

All Python dependencies are pinned in \texttt{requirements.txt}. Key packages
used in experiments are:
\texttt{torch}, \texttt{numpy}, \texttt{pandas}, \texttt{scipy},
\texttt{matplotlib}, \texttt{yfinance}, and \texttt{pyarrow}.

\subsection{Environment Setup}

\begin{verbatim}
python -m venv .venv
# Windows PowerShell
.venv\Scripts\Activate.ps1
# macOS/Linux
source .venv/bin/activate
pip install -r requirements.txt
\end{verbatim}

\section{Data, Universe, and Splits}

Data is downloaded with \texttt{yfinance} through
\texttt{portfolio\_rl.data.YahooReturnsDataset}. Price data is converted to
daily simple returns via percentage change and cached to
\texttt{returns.parquet}.

\begin{itemize}
    \item Asset universe (all default run scripts): \{MSFT, JPM, JNJ, XOM, PG\}.
    \item Date range: 2022-01-01 to 2024-12-31.
    \item Price field: \texttt{Close}.
    \item Train/validation/test split:
    \begin{itemize}
        \item train: dates $\le$ 2023-03-24
        \item validation: 2023-03-25 to 2023-09-30
        \item test: dates $\ge$ 2023-10-01
    \end{itemize}
\end{itemize}

Expanding covariance matrices are computed per split with optional diagonal
projection for training.

\section{Model and Training Implementation}

\subsection{Policy, Value, and Dynamics Networks}

\begin{itemize}
    \item Policy network (\texttt{PolicyNet}): MLP with two hidden layers and
    ReLU activations, outputting logits for a softmax portfolio.
    \item Value network (\texttt{ValueNet}): MLP with two hidden layers and a
    scalar output.
    \item Dynamics model (\texttt{DynamicsModel}): MLP on concatenated
    state-action pairs, predicting Gaussian mean and diagonal variance of the
    next state.
\end{itemize}

The hidden width is controlled by script-level settings (e.g.,
\texttt{networksize} in \texttt{main.py} and \texttt{run\_grid\_search.py}).

\subsection{Kalman Filter}

The filter implementation (\texttt{LearnableKalman}) uses:
\begin{itemize}
    \item fixed transition matrix $A = I$;
    \item diagonal process and measurement noise ($Q$, $R$);
    \item either learned log-diagonal parameters or fixed values provided in
    ablations.
\end{itemize}

\subsection{Reward and Objective}

At each step, the Markowitz-style component is:
\[
r^{\text{mark}}_t
= w_{t-1}^\top z_t
- \lambda \, w_{t-1}^\top \Sigma_t w_{t-1}.
\]

Optionally, uncertainty, turnover cost, and turnover-shaping terms are added:
\[
r_t
= r^{\text{mark}}_t
- \kappa_{\text{unc}} \cdot \tfrac{1}{2}\log\det(\Sigma^{\text{unc}}_t)
- c \lVert w_t - w_{t-1} \rVert_1
+ \beta \exp\!\left(-\tfrac{3}{2}
\left(\frac{\lVert w_t - w_{t-1} \rVert_1}{\tau} - 1\right)^2\right).
\]

Training combines actor, critic, Kalman, dynamics, and simulated policy losses
with scalar weights from \texttt{TrainConfig}.

\subsection{Training Phases}

The main training loop in \texttt{portfolio\_rl/backtest\_seed.py} uses:
\begin{itemize}
    \item 100 epochs KF warmup (\texttt{warmup\_kf}),
    \item 100 epochs dynamics warmup (\texttt{warmup\_dyn}),
    \item policy training for \texttt{cfg.updates} epochs (\texttt{train}).
\end{itemize}

Windows are iterated with
\texttt{T = window\_size + 5} and \texttt{stride = T}. Validation Sharpe is
evaluated periodically using \texttt{cfg.print\_every}.

\section{Evaluation Protocol}

The full-run evaluator computes portfolio returns from lagged weights and
reports:
\begin{itemize}
    \item Sharpe ratio:
    $\text{Sharpe} = \frac{\bar{r}}{s_r + 10^{-8}}\sqrt{252}$,
    \item total return proxy: $1 + \sum_t r_t$,
    \item mean and standard deviation of per-step returns,
    \item maximum drawdown from cumulative equity.
\end{itemize}

During evaluation, policy output is converted to long-only weights via softmax,
with a small $\epsilon$-smoothing and renormalization step.

\section{Determinism and Parallel Execution}

For each run, seeds are set for Python, NumPy, and PyTorch using
\texttt{set\_seed(seed)}. Parallel workers enforce one-thread execution via:
\texttt{torch.set\_num\_threads(1)},
\texttt{OMP\_NUM\_THREADS=1}, and
\texttt{MKL\_NUM\_THREADS=1}.

In spite of seed control, exact bitwise reproducibility is not guaranteed
across machines or library versions because of floating-point and backend
differences.

\section{Default Script Configurations}

The defaults currently in the repository are:
\begin{itemize}
    \item \texttt{main.py}: single run (seed 0), \texttt{window\_size=1},
    \texttt{lam=10}, \texttt{networksize=128}, \texttt{learnrate=1e-5},
    \texttt{updates=1500}, with optional best-checkpoint saving.
    \item \texttt{run\_grid\_search.py}: seeds 0--9, currently configured with
    one-point grids for window size, lambda, network size, and learning rate.
    \item \texttt{run\_backtest\_grid.py}: seeds 0--20, currently configured
    with one-point grids for window size and lambda.
    \item \texttt{sweep\_grid.py}: configurable wider search with append-safe
    CSV logging in \texttt{sweep\_results.csv}.
\end{itemize}

These scripts are intentionally parameterized directly in code. Reproducing
specific thesis figures should therefore use the exact committed values from
the corresponding experiment revision.

\section{Artifacts and Result Files}

Typical generated files are:
\begin{itemize}
    \item \texttt{returns.parquet}: cached returns data.
    \item \texttt{learning\_stats\_kf.csv}: per-epoch training statistics.
    \item \texttt{grid\_search\_lines.txt}: per-job grid log.
    \item \texttt{grid\_search\_summary.txt}: aggregated grid summary.
    \item \texttt{backtest\_grid.txt}: aggregated backtest summary.
    \item \texttt{weights.txt}: latest saved test weights.
    \item \texttt{grid\_search\_raw.csv}, \texttt{grid\_search\_seed\_avg.csv},
    \texttt{grid\_search\_lambda\_avg.csv},
    \texttt{grid\_search\_window\_avg.csv},
    \texttt{grid\_search\_networksize\_avg.csv},
    \texttt{grid\_search\_learnrate\_avg.csv}.
\end{itemize}

The report CSVs are produced by:
\begin{verbatim}
python grid_search_report.py
\end{verbatim}

\section{Minimal Reproduction Procedure}

\begin{enumerate}
    \item Create and activate a clean Python environment.
    \item Install dependencies from \texttt{requirements.txt}.
    \item Run one of:
    \begin{verbatim}
python main.py
python run_grid_search.py
python run_backtest_grid.py
\end{verbatim}
    \item For grid-search post-processing:
    \begin{verbatim}
python grid_search_report.py
\end{verbatim}
    \item For plots:
    \begin{verbatim}
python plot_learning_stats.py learning_stats_kf.csv --out-dir .
\end{verbatim}
\end{enumerate}

This workflow reproduces the full training, evaluation, and artifact-generation
pipeline implemented in the repository.
